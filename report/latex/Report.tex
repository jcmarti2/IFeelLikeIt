\title{Final Project}
\author{Juan Carlos Martinez Mori \\
		Paul Gharzouzi \\
        Jimmy Chang}

\documentclass[11pt,a4paper]{article}
\usepackage{commath}
\usepackage{graphicx}
\usepackage[T1]{fontenc}
\usepackage[textwidth=16cm,textheight=24cm]{geometry}
\usepackage{listings}
\usepackage{blindtext}
\usepackage{float,lscape}
\usepackage{color}
\usepackage{booktabs}
\usepackage{bm}
\definecolor{green}{RGB}{34,139,34}
\definecolor{black}{RGB}{0,0,0}
\definecolor{blue}{RGB}{0,0,255}
\lstset{
  frame=tb,
  language=Python,
  aboveskip=3mm,
  belowskip=3mm,
  showstringspaces=false,
  columns=flexible,
  basicstyle={\small\ttfamily},
  numbers=none,
  numberstyle=\tiny\color{black},
  keywordstyle=\color{blue},
  commentstyle=\color{green},
  stringstyle=\color{blue},
  breaklines=true,
  breakatwhitespace=true,
  tabsize=4,
  title=\lstname
}

\begin{document}
\graphicspath{ {../figures/} }

\maketitle

\section{Project Overview}
This is the project
\section{Infrastructure Interdependence Analysis}

\subsection*{Question 1}
Given
\begin{align}
	x_i = o_i + f_i = \sum_{j} x_{ij} + f_i
\end{align}
\begin{align}
	x_{ij} = a_{ij}x_{j}
\end{align}
we obtain:
\begin{align*}
	x_i = \sum_{j} a_{ij}x_{j} + f_i = a_{i}\bm{x} + f_i,
\end{align*}
where $a_i$ is a $1 \times i$ matrix and $x$ is an $i \times 1$ vector. Similarly, for all cases of $i$, we obtain the matrix equation:
\begin{align} \label{eq: EIO Equation}
	x = Ax + f,
\end{align}
where $x$ is an $i \times 1$ vector, $A$ is a $i \times j$ matrix and $f$ is a $i \times 1$ vector. Note that A must be a square matrix, so $j=i$; its dimensions are $i \times i$.

\subsection*{Question 2}
Table 2 in the given instructions sheet presents matrix $A$, which is the matrix of influence coefficients $a_{ij}$. These coefficients should be understood as the fraction of inoperability transmitted by the $j$th infrastructure to the $i$th infrastructure. \\
\\
The last row of matrix $A$ corresponds to the $i = 10$ infrastructure: satellite communication and navigation. Thereby, we must understand each coefficient $a_{10j}$ for all $j$ to be the fraction of inoperability transmitted by the $j$th infrastructure to the satellite communication and navigation infrastructure (10th).\\
\\
We observe that the coefficients $a_{10j}$ for all $j$ are $0$. This means that the failure of any $j$ infrastructure does not transmit inoperability to the satellite communication and navigation infrastructure. On the other hand, all of the coefficients $a_{i10}$ for all $i \neq 10$ are nonzero. In other words, the operability of the satellite communication and navigation infrastructure is independent of the operability of the other infrastructure, while the operability of the other infrastructure is dependent on the operability of the satellite and communication infrastructure. \\
\\
This assumption seems to be reasonable for a $6-12$ hour outage. One can expect satellites to be self-sufficient in terms of energy consumption and maneuverability, but the infrastructure on the Earth to rely heavily on the data provided by the satellite and communication systems. A satellite may be able to operate on its own during a $6-12$ hour outage of the other infrastructure, while the remaining infrastructure is likely to fail during a $6-12$ hour outage of the satellite and communication infrastructure.

\subsection*{Question 3}
The dependency index of infrastructure $i$, $\gamma_i$ is defined as:
\begin{align}
	\gamma_i = \frac{1}{n-1}\sum_{j \neq i} a_{ij} \text{ (\textit{row summation})}.
\end{align}
The sum of the $a_{ij}$ coefficients reveals the total direct damage on infrastructure $i$ transmitted from the damage of each infrastructure $j$ such that $j \neq i$. By dividing the sum by $n-1$ we compute the index $\gamma_i$, which indicates the average damage on infrastructure $i$ from any other infrastructure. \\
\\
In a sense, this index is a measure of the dependence of an infrastructure on the operability of other infrastructure, where a high value indicates a high dependency and a low value indicates a low dependency. The larger the index, the more direct damage infrastructure $i$ receives from the inoperability of the entire infrastructure system.\\
\\
Likewise, the influence index of infrastructure $j$, $\delta_j$ is defined as:
\begin{align}
	\delta_j = \frac{1}{n-1}\sum_{i \neq j} a_{ij} \text{ (\textit{column summation})}.
\end{align}
The sum of the $a_{ij}$ coefficients reveals the total direct damage of infrastructure $j$ transmitted to the damage of each infrastructure $i$ such that $i \neq j$. By dividing the sum by $n-1$ we compute the index $\delta_j$, which indicates the average influence of infrastructure $j$ has on any other infrastructure. \\
\\
In a sense, this index is a measure of the influence an infrastructure has on the operability of other infrastructure, where a high value indicates a high influence and a low value indicates a low influence. The larger the index, the greater the direct impact the inoperability of infrastructure $j$ has on the entire infrastructure system. \\
\\

\subsection*{Question 4}
Starting from Equation \ref{eq: EIO Equation}, we can compute the following:
\begin{align*}
	x &= Ax + f \\
	Ix &= Ax + f \\
	(I-A)x &= f \\
	x &= (I-A)^{-1}f \\
\end{align*}
We can express the matrix $(I-A)^{-1}$ as matrix $S$, finally obtaining the solution in the form of:
\begin{align} \label{eq: EIO Solution}
	x = Sf
\end{align}
Note that the information provided in Table 2 corresponds to matrix $A$. To compute the $S$ in Equation \ref{eq: EIO Solution} we need to follow our definition of S, $S = (I-A)^{-1}$. The matrix $A$ must be a square matrix with coefficients between $0$ and $1$ and the $I-A$ matrix must invertible. \\
\\
Both matrix $A$ and matrix $S$ are composed of $i$ rows and $j$ columns, where $i=j$. The elements of matrix $A$, $a_{ij}$, represent the first damage propagation step from infrastructure $j$ on infrastructure $i$. The elements of matrix $S$, $s_{ij}$, represent the total damage propagation from infrastructure $j$ on infrastructure $i$. This is, the total damage propagation from infrastructure $j$ on infrastructure $i$ that the system converges to after a series of recursive damage propagation steps.

\subsection*{Question 5}
The matrix $S$ is an indicator of the total impact of an external shock on the entire infrastructure system. In particular, every $s_{ij}$ reveals the infinite propagation of the damage of an infrastructure sector $j$ to infrastructure sector $i$. Every $s_{ij}$ should be greater than or equal to the corresponding $a_{ij}$ from matrix $A$. This is because $s_{ij}$ includes the direct damage captured by $a_{ij}$ and the subsequent inoperability propagation of failure, as explained in Question 4. \\
\\
The matrix $S$ computed after the matrix $A$ found in Table 2 of the instructions sheet can be found in Table \ref{tab: Matrix S}.

\begin{table}[H]
  \centering
  \caption{Matrix $S$ for a 6-12 hr outage.}
    \begin{tabular}{r|rrrrrrrrrr}
    \toprule
    Sector Id & \multicolumn{1}{c}{1} & \multicolumn{1}{c}{2} & \multicolumn{1}{c}{3} & \multicolumn{1}{c}{4} & \multicolumn{1}{c}{5} & \multicolumn{1}{c}{6} & \multicolumn{1}{c}{7} & \multicolumn{1}{c}{8} & \multicolumn{1}{c}{9} & \multicolumn{1}{c}{10} \\
    \midrule
    1     & 1.004 & 0.237 & 0.325 & 0.510 & 0.038 & 0.029 & 0.016 & 0.027 & 0.053 & 0.319 \\
    2     & 0.001 & 1.012 & 0.013 & 0.027 & 0.002 & 0.003 & 0.001 & 0.003 & 0.180 & 0.005 \\
    3     & 0.003 & 0.124 & 1.005 & 0.126 & 0.003 & 0.006 & 0.003 & 0.003 & 0.027 & 0.009 \\
    4     & 0.006 & 0.089 & 0.017 & 1.008 & 0.005 & 0.004 & 0.004 & 0.002 & 0.020 & 0.008 \\
    5     & 0.005 & 0.061 & 0.013 & 0.027 & 1.001 & 0.006 & 0.008 & 0.008 & 0.019 & 0.022 \\
    6     & 0.002 & 0.263 & 0.111 & 0.131 & 0.007 & 1.002 & 0.008 & 0.007 & 0.049 & 0.008 \\
    7     & 0.004 & 0.118 & 0.035 & 0.110 & 0.008 & 0.004 & 1.001 & 0.004 & 0.029 & 0.011 \\
    8     & 0.009 & 0.535 & 0.114 & 0.087 & 0.052 & 0.023 & 0.022 & 1.003 & 0.097 & 0.016 \\
    9     & 0.002 & 0.036 & 0.011 & 0.009 & 0.005 & 0.000 & 0.002 & 0.005 & 1.006 & 0.006 \\
    10    & 0.000 & 0.000 & 0.000 & 0.000 & 0.000 & 0.000 & 0.000 & 0.000 & 0.000 & 1.000 \\
    \bottomrule
    \end{tabular}%
  \label{tab: Matrix S}%
\end{table}%
From Table \ref{tab: Matrix S} we can observe that $s_{12} = 0.237$ is smaller than $s_{82}=0.535$. This indicates that sector 2 has a bigger total impact on sector 8 than on sector 1 due to the propagation of damage of sector 2 to the entire infrastructure system. \\
\\
Comparing the case for the elements of matrix $A$, we can observe that $a_{101}$ is 0 and $s_{101}$ remained 0. This means that there is neither direct propagation of inoperability of sector 1 (air transportation) to sector 10 (satellite communication and navigation) during a 6-12 hr outage, nor total damage propagation. \\
\\
However, $a_{22}$ is 0 and the corresponding $s_{22}$ is 1.012. This reveals that there is no direct damage propagation from the infrastructure on itself (based on the given assumption), yet there is a total damage on sector 2 due to the propagation of damage on the entire infrastructure system. 


\subsection*{Question 6}
As explained in Questions 4 and 5, matrix $S$ encompasses matrix $A$. Matrix $A$ captures the direct propagation of inoperability of the entire system whereas matrix $S$ represents the infinite propagation of damage on the system. Therefore, it is easier to collect and to model the first tier of damage for every combination of $ij$ pairs than to model the infinite propagation of damage.\\
\\
From a data collection perspective, Setola et al. explain that "infrastructure owners and operators generally have a good idea of how much their infrastructures depend on the resources provided directly by other infrastructures" (2009). However, assessing further propagation of damage is more challenging because of the limited information that the owners and operators have on the "implications of higher-order (inter)dependencies" (Setola et al., 2009). \\
\\
Furthermore, despite the complexity of modeling matrix $S$, there is a straightforward approach to find $S$ once we have matrix $A$, as shown in Question 1. This further emphasizes on why it is easier to obtain the elements in matrix $A$ compared to the ones in $S$.

\subsection*{Question 7}

The overall dependency index of infrastructure $i$, $\overline\gamma_i$ is defined as:
\begin{align}
	\overline\gamma_i = \frac{1}{n-1}\sum_{j \neq i} s_{ij} \text{ (\textit{row summation})}.
\end{align}
The sum of the $s_{ij}$ coefficients reveals the overall damage on infrastructure $i$ transmitted from the damage of each infrastructure $j$ ??such that $j \neq i$??, as a result of the higher-order interdependencies in the infrastructure system. By dividing the sum by $n-1$ we compute the index $\\overline\gamma_i$, which indicates the average overall damage on infrastructure $i$ from any other infrastructure. \\
\\
In a sense, we define this index as a measure of the vulnerability of an infrastructure sector within the entire infrastructure system, where a high value indicates a high vulnerability and a low value indicates a low vulnerability. The larger the index, the bigger the total damage infrastructure $i$ receives from the inoperability of the entire infrastructure system.\\
\\
Likewise, the influence index of infrastructure $j$, $\overline\delta_j$ is defined as:
\begin{align}
	\overline\delta_j = \frac{1}{n-1}\sum_{i \neq j} s_{ij} \text{ (\textit{column summation})}.
\end{align}
The sum of the $s_{ij}$ coefficients reveals the overall damage transmitted from infrastructure $j$ to each infrastructure $i$ ?? such that $i \neq j$ ??, as a result of the higher-order interdependencies in the infrastructure system. By dividing the sum by $n-1$ we compute the index $\overline\delta_j$, which indicates the average total influence that infrastructure $j$ has on any other infrastructure.\\
\\
In a sense, we define this index as a measure of the criticality of an infrastructure sector within the infrastructure system, where a high value indicates a high criticality and a low value indicates a low criticality. The larger the index, the greater the total impact the inoperability of infrastructure $j$ has on the entire infrastructure system, so the more critical this sector $j$ is.

\subsection*{Question 8}

By comparing the coefficients of column 2 of matrix $A$ for the electricity sector (Table 2 of the instructions sheet), we can observe that electricity has the biggest direct influence on the sector of fuel and petroleum grid ($a_{82} = 0.500$). Moreover, it appears that the electricity sector directly depends the most on the sector of natural gas, based on the coefficients of row 2 of matrix $A$ ($a_{29} = 0.178$).\\
\\
As for the sector of natural gas, we can observe that this infrastructure has the greatest influence on the electricity sector ($a_{29} = 0.178$) while also directly depending on the electricity the most ($a_{92} = 0.030$). These coefficients reveal the significant dependency between the sectors of electricity and natural gas.\\
\\
Besides the observations for each sector, we can see that the column coefficients (column 2)for the electricity sector are generally higher than the row values (row 2). This means that the electricity sector has a larger direct influence on the other sectors compared to its direct dependence on the infrastructure system. The natural gas sector has a low direct dependency, shown by having 8 of 9 row coefficients (excluding itself) be less than 0.01.In addition, the dependency and influence coefficients for the electricity sector are generally higher than the ones for natural gas, which means that the electricity sector is more critical and dependent more by the damage of other sectors. From a practical perspective, these observations can be realistic because the natural gas is a fuel source with other substitutes in the market, whereas electricity has no feasible substitutes in a 6-12 hour outage.

\subsection*{Question 9}

The indices $\gamma_i$ and $\delta_j$ have been calculated using a script and a class in MATLAB. The class includes the equations (5) and (6) provided in the instructions sheet as part of the public functions NAME. These functions are called within the script NAME to compute the indices $\gamma_i$ and $\delta_j$. 
\underline{\textbf{Note:}} For the aforementioned script and class, please refer to the appendix. For the MATLAB code, please refer to the script entitled  and the class entitled  that are uploaded on compass 2g.

The indices $\gamma_i$ and $\delta_j$ for every infrastructure sector are presented in the figures NUMBER AND NUMBER below.

\begin{figure}
	\label{fig: Gamma}
	\centering
	\includegraphics[width=0.8\textwidth]
    {gamma.png}
    \caption{Dependency index for all the sectors}
\end{figure}

\begin{figure}
	\label{fig: Delta}
	\centering
	\includegraphics[width=0.8\textwidth]
    {delta.png}
    \caption{Influence index for all the sectors}
\end{figure}

From figure NUMBER, we observe that the air transportation sector (sector 1) has the highest dependency index ($\gamma_1 = 0.145$). This means that the air transportation sector
receives the biggest direct damage from the inoperability of the entire infrastructure system. The sectors of fuel and petroleum grid and rail transportation are respectively the second and third most dependent infrastructures within the system ($\gamma_8 = 0.084$  and  $\gamma_6 = 0.052$). These indices reveal that the transportation sector in general is highly dependent on other infrastructure sectors within the system and subsequently the first sector to be influenced or damaged. Other than the previous three sectors, the remaining infrastructures have low direct dependency on the inoperability of other sectors as all of the remaining indices have a value below $0.03$. Note that the satellite communication and navigation sector (sector 10) has an index of ($\gamma_10 = 0$) which indicates that sector 10 is not actually affected by the inoperability of any other infrastructure within the system.\\
\\
As for the influence index, we observe that the electricity sector (sector 2) has the highest influence index ($\delta_2 = 0.138$). This means that the electricity sector
transmits the biggest direct damage from its inoperability to the entire infrastructure system. The sectors of TLC wired and TLC wireless are respectively the second and third most influential infrastructures within the system ($\delta_4 = 0.098$  and  $\delta_3 = 0.065$). Furthermore, the satellite communication and navigation sector has a relatively high influence index ($\delta_10 = 0.041$), which reveals that inoperability of sector 10 has a sizable effect on the infrastructure system. These indices reveal that the electricity and the communication sectors in general have the largest direct impact on other infrastructure sectors as a result of their inoperability. Other than the previous four sectors, the remaining infrastructures have low direct influence on the inoperability of other sectors as all of the remaining indices have a value below $0.025$.

\subsection*{Question 10}
Similar to Question 9, the indices $\overline\gamma_i$ and $\overline\delta_j$ have been calculated using a script and a class in MATLAB.
\underline{\textbf{Note:}} For the aforementioned script and class, please refer to the appendix. For the MATLAB code, please refer to the script entitled NAME and the class entitled NAME that are uploaded on compass 2g.

The indices $\overline\gamma_i$ and $\overline\delta_j$ for every infrastructure sector are presented in the figures NUMBER AND NUMBER below.

\begin{figure}
	\label{fig: Gamma bar}
	\centering
	\includegraphics[width=0.8\textwidth]
    {gamma_bar.png}
    \caption{Overall dependency index for all the sectors}
\end{figure}

\begin{figure}
	\label{fig: Delta bar}
	\centering
	\includegraphics[width=0.8\textwidth]
    {delta_bar.png}
    \caption{Overall influence index for all the sectors}
\end{figure}

The comparison of the indices $\gamma_j$ and $\overline\gamma_j$ further stresses on the difference

From figure NUMBER, we observe that the air transportation sector (sector 1) has the highest dependency index ($\gamma_1 = 0.145$). This means that the air transportation sector
receives the biggest direct damage from the inoperability of the entire infrastructure system. The sectors of fuel and petroleum grid and rail transportation are respectively the second and third most dependent infrastructures within the system ($\gamma_8 = 0.084$  and  $\gamma_6 = 0.052$). These indices reveal that the transportation sector in general is highly dependent on other infrastructure sectors within the system and subsequently the first sector to be influenced or damaged. Other than the previous three sectors, the remaining infrastructures have low direct dependency on the inoperability of other sectors as all of the remaining indices have a value below $0.03$. Note that the satellite communication and navigation sector (sector 10) has an index of ($\gamma_10 = 0$) which indicates that sector 10 is not actually affected by the inoperability of any other infrastructure within the system.\\
\\
As for the influence index, we observe that the electricity sector (sector 2) has the highest influence index ($\delta_2 = 0.138$). This means that the electricity sector
transmits the biggest direct damage from its inoperability to the entire infrastructure system. The sectors of TLC wired and TLC wireless are respectively the second and third most influential infrastructures within the system ($\delta_4 = 0.098$  and  $\delta_3 = 0.065$). Furthermore, the satellite communication and navigation sector has a relatively high influence index ($\delta_10 = 0.041$), which reveals that inoperability of sector 10 has a sizable effect on the infrastructure system. These indices reveal that the electricity and the communication sectors in general have the largest direct impact on other infrastructure sectors as a result of their inoperability. Other than the previous four sectors, the remaining infrastructures have low direct influence on the inoperability of other sectors as all of the remaining indices have a value below $0.025$.

Add the S and bigger A
and then refer to upcoming questions in with the tier stuff.

Sector 10 doesn't have an increase because it is already independent.

\subsection*{Question 11}
\begin{itemize}
	\item How far reaching is this smart grid? How much effect does it have? From washing machines to cell phone use
	\item DAQ
\end{itemize}

\subsection*{Question 12}
\begin{itemize}
	\item TODO: Matlab code to add/decrease 10%
\end{itemize}

\subsection*{Question 13}
\begin{itemize}
	\item solve x = Sf
	\item TODO: Matlab code
	\item Effect on other infrastructure assets: check vector x
	\item Further degrade of the already damaged infrastructure: compare x of the infrastructure to f (see if effects increase or decrease)
	\item Other assets degreaded indirectly: Yes, by having matrix S
\end{itemize}

\begin{itemize}
	\item The effect on other infrastructures is given through the matrix $x$.
	\item Critical infrastructures further degrade the air transportation sector by 3.96 percent.
	\item Critical infrastructures further degrade the rail transportation sector by 2.81 percent.
	\item Critical infrastructures further degrade the electricity sector by 0.30 percent.
	\item Other infrastructures assets are degraded even without being directly affected by the storm. The sector that gets affected the most without a direct damage due to the storm is the fuel and petroleum grid which is degraded by 6.83 percent.
	\item There are sectors that had little to no damage due to the storm as well. The satellite communication and navigation sector was not influenced by the storm at all while natural gas was only degraded by 0.45 percent.
\end{itemize}

\subsection*{Question 14}
\begin{itemize}
	\item No
\end{itemize}

\subsection*{Question 15}
\begin{itemize}
	\item Recursion code MATLAB TODO
	\item paul took picture
\end{itemize}

\subsection*{Question 16}
\begin{itemize}
	\item monte carlo UNCERTAINTY ON A
\end{itemize}

\subsection*{Question 17}
\begin{itemize}
	\item monte carlo UNCERTAINTY ON F
\end{itemize}

\subsection*{Question 18}
\begin{itemize}
	\item HIGH LOW METHOD
	\item write function that adds noise with certain g
\end{itemize}

\subsection*{Question 19}

\end{document}
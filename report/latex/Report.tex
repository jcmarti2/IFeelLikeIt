\title{Final Project}
\author{Juan Carlos Martinez Mori \\
		Paul Gharzouzi \\
        Jimmy Chang}

\documentclass[11pt,a4paper]{article}
\usepackage{commath}
\usepackage{graphicx}
\usepackage[T1]{fontenc}
\usepackage[textwidth=16cm,textheight=24cm]{geometry}
\usepackage{listings}
\usepackage{blindtext}
\usepackage{float,lscape}
\usepackage{color}
\usepackage{booktabs}
\usepackage{bm}
\definecolor{green}{RGB}{34,139,34}
\definecolor{black}{RGB}{0,0,0}
\definecolor{blue}{RGB}{0,0,255}
\lstset{
  frame=tb,
  language=Python,
  aboveskip=3mm,
  belowskip=3mm,
  showstringspaces=false,
  columns=flexible,
  basicstyle={\small\ttfamily},
  numbers=none,
  numberstyle=\tiny\color{black},
  keywordstyle=\color{blue},
  commentstyle=\color{green},
  stringstyle=\color{blue},
  breaklines=true,
  breakatwhitespace=true,
  tabsize=4,
  title=\lstname
}

\begin{document}
\graphicspath{ {../figures/} }

\maketitle

\section{Project Overview}
This is the project
\section{Infrastructure Interdependence Analysis}

\subsection*{Question 1}
Given
\begin{align}
	x_i = o_i + f_i = \sum_{j} x_{ij} + f_i
\end{align}
\begin{align}
	x_{ij} = a_{ij}x_{j}
\end{align}
we obtain:
\begin{align*}
	x_i = \sum_{j} a_{ij}x_{j} + f_i = a_{i}\bm{x} + f_i,
\end{align*}
where $a_i$ is a $1 \times i$ matrix and $x$ is an $i \times 1$ vector. Similarly, for all cases of $i$, we obtain the matrix equation:
\begin{align} \label{eq: EIO Equation}
	x = Ax + f,
\end{align}
where $x$ is an $i \times 1$ vector, $A$ is a $i \times j$ matrix and $f$ is a $i \times 1$ vector. Note that A must be a square matrix, so $j=i$; its dimensions are $i \times i$.

\subsection*{Question 2}
Table 2 in the given instructions sheet presents matrix $A$, which is the matrix of influence coefficients $a_{ij}$. These coefficients should be understood as the fraction of inoperability transmitted by the $j$th infrastructure to the $i$th infrastructure. \\
\\
The last row of matrix $A$ corresponds to the $i = 10$ infrastructure: satellite communication and navigation. Thereby, we must understand each coefficient $a_{10j}$ for all $j$ to be the fraction of inoperability transmitted by the $j$th infrastructure to the satellite communication and navigation infrastructure (10th).\\
\\
We observe that the coefficients $a_{10j}$ for all $j$ are $0$. This means that the failure of any $j$ infrastructure does not transmit inoperability to the satellite communication and navigation infrastructure. On the other hand, all of the coefficients $a_{i10}$ for all $i \neq 10$ are nonzero. In other words, the operability of the satellite communication and navigation infrastructure is independent of the operability of the other infrastructure, while the operability of the other infrastructure is dependent on the operability of the satellite and communication infrastructure. \\
\\
This assumption seems to be reasonable for a $6-12$ hour outage. One can expect satellites to be self-sufficient in terms of energy consumption and maneuverability, but the infrastructure on the Earth to rely heavily on the data provided by the satellite and communication systems. A satellite may be able to operate on its own during a $6-12$ hour outage of the other infrastructure, while the remaining infrastructure is likely to fail during a $6-12$ hour outage of the satellite and communication infrastructure.

\subsection*{Question 3}
The dependency index of infrastructure $i$, $\gamma_i$ is defined as:
\begin{align}
	\gamma_i = \frac{1}{n-1}\sum_{j \neq i} a_{ij} \text{ (\textit{row summation})}.
\end{align}
The sum of the $a_{ij}$ coefficients reveals the total direct damage on infrastructure $i$ transmitted from the damage of each infrastructure $j$ such that $j \neq i$. By dividing the sum by $n-1$ we compute the index $\gamma_i$, which indicates the average damage on infrastructure $i$ from any other infrastructure. \\
\\
In a sense, this index is a measure of the dependence of an infrastructure on the operability of other infrastructure, where a high value indicates a high dependency and a low value indicates a low dependency. The larger the index, the more direct damage infrastructure $i$ receives from the inoperability of the entire infrastructure system.\\
\\
Likewise, the influence index of infrastructure $j$, $\delta_j$ is defined as:
\begin{align}
	\delta_j = \frac{1}{n-1}\sum_{i \neq j} a_{ij} \text{ (\textit{column summation})}.
\end{align}
The sum of the $a_{ij}$ coefficients reveals the total direct damage of infrastructure $j$ transmitted to the damage of each infrastructure $i$ such that $i \neq j$. By dividing the sum by $n-1$ we compute the index $\delta_j$, which indicates the average influence of infrastructure $j$ has on any other infrastructure. \\
\\
In a sense, this index is a measure of the influence an infrastructure has on the operability of other infrastructure, where a high value indicates a high influence and a low value indicates a low influence. The larger the index, the larger direct impact from the inoperability of infrastructure $j$ on the entire infrastructure system. \\
\\

\subsection*{Question 4}
Starting from Equation \ref{eq: EIO Equation}, we can compute the following:
\begin{align*}
	x &= Ax + f \\
	Ix &= Ax + f \\
	(I-A)x &= f \\
	x &= (I-A)^{-1}f \\
\end{align*}
We can express the matrix $(I-A)^{-1}$ as matrix $S$, finally obtaining the solution in the form of:
\begin{align} \label{eq: EIO Solution}
	x = Sf
\end{align}
Note that the information provided in Table 2 corresponds to matrix $A$. To compute the $S$ in Equation \ref{eq: EIO Solution} we need to follow our definition of S, $S = (I-A)^{-1}$. The matrix $A$ must be a square matrix with coefficients between $0$ and $1$ and the $I-A$ matrix must invertible. \\
\\
Both matrix $A$ and matrix $S$ are composed of $i$ rows and $j$ columns, where $i=j$. The elements of matrix $A$, $a_{ij}$, represent the first damage propagation step from infrastructure $j$ on infrastructure $i$. The elements of matrix $S$, $s_{ij}$, represent the total damage propagation from infrastructure $j$ on infrastructure $i$. This is, the total damage propagation from infrastructure $j$ on infrastructure $i$ that the system converges to after a series of recursive damage propagation steps.

\subsection*{Question 5}
The matrix $S$ is an indicator of the total impact of an external shock on the entire infrastructure system. In particular, every $s_{ij}$ reveals the infinite propagation of the damage of an infrastructure sector $j$ to infrastructure sector $i$. Every $s_{ij}$ should be greater than or equal to the corresponding $a_{ij}$ from matrix $A$. This is because $s_{ij}$ includes the direct damage captured by $a_{ij}$ and the subsequent inoperability propagation of failure, as explained in Question 4. \\
\\
The matrix $S$ computed after the matrix $A$ found in Table 2 of the instructions sheet can be found in Table \ref{tab: Matrix S}.

\begin{table}[H]
  \centering
  \caption{Matrix $S$ for a 6-12 hr outage.}
    \begin{tabular}{r|rrrrrrrrrr}
    \toprule
    Sector Id & \multicolumn{1}{c}{1} & \multicolumn{1}{c}{2} & \multicolumn{1}{c}{3} & \multicolumn{1}{c}{4} & \multicolumn{1}{c}{5} & \multicolumn{1}{c}{6} & \multicolumn{1}{c}{7} & \multicolumn{1}{c}{8} & \multicolumn{1}{c}{9} & \multicolumn{1}{c}{10} \\
    \midrule
    1     & 1.004 & 0.237 & 0.325 & 0.510 & 0.038 & 0.029 & 0.016 & 0.027 & 0.053 & 0.319 \\
    2     & 0.001 & 1.012 & 0.013 & 0.027 & 0.002 & 0.003 & 0.001 & 0.003 & 0.180 & 0.005 \\
    3     & 0.003 & 0.124 & 1.005 & 0.126 & 0.003 & 0.006 & 0.003 & 0.003 & 0.027 & 0.009 \\
    4     & 0.006 & 0.089 & 0.017 & 1.008 & 0.005 & 0.004 & 0.004 & 0.002 & 0.020 & 0.008 \\
    5     & 0.005 & 0.061 & 0.013 & 0.027 & 1.001 & 0.006 & 0.008 & 0.008 & 0.019 & 0.022 \\
    6     & 0.002 & 0.263 & 0.111 & 0.131 & 0.007 & 1.002 & 0.008 & 0.007 & 0.049 & 0.008 \\
    7     & 0.004 & 0.118 & 0.035 & 0.110 & 0.008 & 0.004 & 1.001 & 0.004 & 0.029 & 0.011 \\
    8     & 0.009 & 0.535 & 0.114 & 0.087 & 0.052 & 0.023 & 0.022 & 1.003 & 0.097 & 0.016 \\
    9     & 0.002 & 0.036 & 0.011 & 0.009 & 0.005 & 0.000 & 0.002 & 0.005 & 1.006 & 0.006 \\
    10    & 0.000 & 0.000 & 0.000 & 0.000 & 0.000 & 0.000 & 0.000 & 0.000 & 0.000 & 1.000 \\
    \bottomrule
    \end{tabular}%
  \label{tab: Matrix S}%
\end{table}%
From Table \ref{tab: Matrix S} we can observe that $s_{12} = 0.237$ is smaller than $s_{82}=0.535$. This indicates that sector 2 has a bigger total impact on sector 8 than on sector 1 due to the propagation of damage of sector 2 to the entire infrastructure system. \\
\\
Comparing the case for the elements of matrix $A$, we can observe that $a_{101}$ is 0 and $s_{101}$ remained 0. This means that there is neither direct propagation of inoperability of sector 1 (air transportation) to sector 10 (satellite communication and navigation) during a 6-12 hr outage, nor total damage propagation. \\
\\
However, $a_{22}$ is 0 and the corresponding $s_{22}$ is 1.012. This reveals that there is no direct damage propagation from the infrastructure on itself (based on the given assumption), yet there is a total damage on sector 2 due to the propagation of damage on the entire infrastructure system. 


\subsection*{Question 6}
As explained in Questions 4 and 5, matrix $S$ encompasses matrix $A$. Matrix $A$ captures the direct propagation of inoperability of the entire system whereas matrix $S$ represents the infinite propagation of damage on the system. Therefore, it is easier to collect and to model the first tier of damage for every combination of $ij$ pairs than to model the infinite propagation of damage.\\
\\
\begin{itemize}
	\item Check of paper of Setola
	\item Dan said that A is computed from surveys, and it is easier to tell how your own infrastructe would be affected by failure of other infrastructure than
	\item From a data collection perspective it is easier to calculate the direct impact of a failure of one infrastructure to another than the propagation of failures from one infrastructure to the rest
	\item From a modeling perspective, it is easier to keep track of matrix A as the direct dependencies between sectors are more intuitive than the chain propgatations between all of the sectors. If errors arise, it is easier to check what went wrong with A than trying to figure out what is wrong with S. 
\end{itemize}

\subsection*{Question 7}
\begin{itemize}
	\item Write down equations (replace a ij with s ij)
	\item The overall influence index in a way is the same definition but instead of direct is total. At the end, this index reveals the most critical sectors of infrastructure system
	\item The overall dependency index is the same definition but instead is total. This index reveals the sectors that are most dependent on the operability of the other sectors of the infrastructure system
	\item You are as strong as your weaskest link 
\end{itemize}

\subsection*{Question 8}
\begin{itemize}
	\item Electricity is most dependent on natural gas while natural gas is most dependent on electricity
	\item Electricity influecnes fuel and petroleum the most while natural gas influences electricty the msot
	\item For sector 2, we see that the column values are generally higher than the row values, which means that sector 2 has a great direct influence on the other sectors while is less impacteed directly by the failure of other sectors
	\item Similarly for natural gas, except that 8 of the 9 row coefficients (excluding index 9,9) are below 0.01, which reveals very low direct dependency of the natural gas sector. Also the column of natural gas is 
	\item In both cases, it looks like electricty is more critical and is influenced more by the damage of other sectors
	\item THis makes sense, because the natural gas is a fuel source with substitutes, while electricity has really no substitutes
\end{itemize}

\subsection*{Question 9}
\begin{itemize}
	\item TODO: MATLAB code PAUL
	\item Plot
	\item Discuss after code is done and put answer
\end{itemize}


\subsection*{Question 10}
\begin{itemize}
	\item TODO: MATLAB code pending
	\item plot
\end{itemize}

\subsection*{Question 11}
\begin{itemize}
	\item How far reaching is this smart grid? How much effect does it have? From washing machines to cell phone use
	\item DAQ
\end{itemize}

\subsection*{Question 12}
\begin{itemize}
	\item TODO: Matlab code to add/decrease 10%
\end{itemize}

\subsection*{Question 13}
\begin{itemize}
	\item solve x = Sf
	\item TODO: Matlab code
	\item Effect on other infrastructure assets: check vector x
	\item Further degrade of the already damaged infrastructure: compare x of the infrastructure to f (see if effects increase or decrease)
	\item Other assets degreaded indirectly: Yes, by having matrix S
\end{itemize}

\begin{itemize}
	\item The effect on other infrastructures is given through the matrix $x$.
	\item Critical infrastructures further degrade the air transportation sector by 3.96 percent.
	\item Critical infrastructures further degrade the rail transportation sector by 2.81 percent.
	\item Critical infrastructures further degrade the electricity sector by 0.30 percent.
	\item Other infrastructures assets are degraded even without being directly affected by the storm. The sector that gets affected the most without a direct damage due to the storm is the fuel and petroleum grid which is degraded by 6.83 percent.
	\item There are sectors that had little to no damage due to the storm as well. The satellite communication and navigation sector was not influenced by the storm at all while natural gas was only degraded by 0.45 percent.
\end{itemize}

\subsection*{Question 14}
\begin{itemize}
	\item No
\end{itemize}

\subsection*{Question 15}
\begin{itemize}
	\item Recursion code MATLAB TODO
	\item paul took picture
\end{itemize}

\subsection*{Question 16}
\begin{itemize}
	\item monte carlo UNCERTAINTY ON A
\end{itemize}

\subsection*{Question 17}
\begin{itemize}
	\item monte carlo UNCERTAINTY ON F
\end{itemize}

\subsection*{Question 18}
\begin{itemize}
	\item HIGH LOW METHOD
	\item write function that adds noise with certain g
\end{itemize}

\subsection*{Question 19}

\end{document}